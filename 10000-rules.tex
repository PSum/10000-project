\documentclass[a4paper,10pt]{article}
\usepackage[margin=5mm,noheadfoot]{geometry} % Keine Ränder, keine Kopf-/Fußzeilen
\usepackage{float}
\usepackage{booktabs}
\pagestyle{empty} % Keine Seitenzahlen
\newlength{\adjustedWidth}
\setlength{\adjustedWidth}{42.5mm}

\setlength{\textwidth}{\adjustedWidth}
\usepackage{tabularx}
\setlength{\oddsidemargin}{0pt}
\setlength{\evensidemargin}{0pt}
\setlength{\parindent}{0pt}
\renewcommand{\familydefault}{\sfdefault}
\usepackage{microtype}          % Schönere Buchstabenabstände

\usepackage[ngerman]{babel} % Für Silbentrennung
\usepackage{enumitem}       % Feinsteuerung der Listen


% Itemize-Einrückung minimieren
\setlist[itemize]{leftmargin=*,labelsep=0.5em,itemsep=5pt,parsep=0pt}

\usepackage{tikz}
\newcommand{\di}[1]{%
\begin{tikzpicture}[scale=0.3,baseline=-0.5ex]
\draw (0,0) rectangle (1,1);
\ifnum#1=1
    \fill (0.5,0.5) circle (0.1);
\fi
\ifnum#1=2
    \fill (0.25,0.25) circle (0.1);
    \fill (0.75,0.75) circle (0.1);
\fi
\ifnum#1=3
    \fill (0.25,0.25) circle (0.1);
    \fill (0.5,0.5) circle (0.1);
    \fill (0.75,0.75) circle (0.1);
\fi
\ifnum#1=4
    \fill (0.25,0.25) circle (0.1);
    \fill (0.25,0.75) circle (0.1);
    \fill (0.75,0.25) circle (0.1);
    \fill (0.75,0.75) circle (0.1);
\fi
\ifnum#1=5
    \fill (0.25,0.25) circle (0.1);
    \fill (0.25,0.75) circle (0.1);
    \fill (0.5,0.5) circle (0.1);
    \fill (0.75,0.25) circle (0.1);
    \fill (0.75,0.75) circle (0.1);
\fi
\ifnum#1=6
    \fill (0.25,0.25) circle (0.1);
    \fill (0.25,0.5)  circle (0.1);
    \fill (0.25,0.75) circle (0.1);
    \fill (0.75,0.25) circle (0.1);
    \fill (0.75,0.5)  circle (0.1);
    \fill (0.75,0.75) circle (0.1);
\fi
\end{tikzpicture}%
}

\sloppy % Lockerer Umbruch, verhindert Überstand

\begin{document}
\begin{itemize}
\item Es wird mit insgesamt sechs Würfeln gespielt.
\item Ziel des Spiels ist, in möglichst wenigen Runden auf zehntausend Punkte zu kommen.
\item Eine Runde beginnt damit, dass mit allen sechs Würfeln der erste Wurf erfolgt.
\item Nach jedem Wurf muss mindestens ein Punktewürfel weggelegt werden.
\item Die Wertigkeit der einzelnen Würfel und Würfelkombinationen steht in der Tabelle weiter unten.
\item Sind alle Würfel in einem oder mehreren Würfen erfolgreich weggelegt worden, so kann die Runde (wieder mit allen sechs Würfeln) fortgesetzt werden.
\item Eine Runde ist dann beendet, wenn entweder die gesammelten Punkte geschrieben, oder bei einem Wurf keine Punkte erzielt wurden.
\item Es darf erst ab dreihundertfünfzig Punkten geschrieben werden.
\item Wird ein Pasch oder eine Straße geworfen, so muss dies mit mindestens dreihundertfünfzig Punkten bestätigt werden. Eher dürfen die gesammelten Punkte nicht geschrieben werden.
\end{itemize}
% Table generated by Excel2LaTeX from sheet 'Sheet1'
\begin{table}[H]
  \centering
    \begin{tabularx}{\adjustedWidth}{X}
    \toprule
    \di{5} = 50 Punkte \\
    \di{1} = 100 Punkte \\
    \midrule
    3 x \di{2} = 200 Punkte \\
    3 x \di{3} = 300 Punkte \\
    3 x \di{4} = 400 Punkte \\
    3 x \di{5} = 500 Punkte \\
    3 x \di{6} = 600 Punkte \\
    3 x \di{1} = 1000 Punkte \\
    \midrule
    Jeder weitere Würfel:\\
     + 1000 Punkte \\
    zum Beispiel: \\
    4 x \di{5} = 1500 Punkte \\
    5 x \di{3} = 2300 Punkte \\
    6 x \di{1} = 4000 Punkte \\
    \midrule
    Pasch (3 Pärchen) oder\\
    z.B.: \di{1} \di{1} \di{2} \di{2} \di{4} \di{4}\\
    Straße (1 bis 6)\\
    z.B.: \di{1} \di{2} \di{3} \di{4} \di{5} \di{6}\\
    = 1500 Punkte\\
    \bottomrule
    \end{tabularx}%
\end{table}%

\end{document}
