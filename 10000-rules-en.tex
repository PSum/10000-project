\documentclass[a4paper,10pt]{article}
\usepackage[margin=5mm,noheadfoot]{geometry} % Keine Ränder, keine Kopf-/Fußzeilen
\usepackage{float}
\usepackage{booktabs}
\pagestyle{empty} % Keine Seitenzahlen
\newlength{\adjustedWidth}
\setlength{\adjustedWidth}{45mm}

\setlength{\textwidth}{\adjustedWidth}
\usepackage{tabularx}
\setlength{\oddsidemargin}{0pt}
\setlength{\evensidemargin}{0pt}
\setlength{\parindent}{0pt}
\renewcommand{\familydefault}{\sfdefault}
\usepackage{microtype}          % Schönere Buchstabenabstände

\usepackage[ngerman]{babel} % Für Silbentrennung
\usepackage{enumitem}       % Feinsteuerung der Listen


% Itemize-Einrückung minimieren
\setlist[itemize]{leftmargin=*,labelsep=0.5em,itemsep=5pt,parsep=0pt}

\usepackage{tikz}
\newcommand{\di}[1]{%
\begin{tikzpicture}[scale=0.3,baseline=-0.5ex]
\draw (0,0) rectangle (1,1);
\ifnum#1=1
    \fill (0.5,0.5) circle (0.1);
\fi
\ifnum#1=2
    \fill (0.25,0.25) circle (0.1);
    \fill (0.75,0.75) circle (0.1);
\fi
\ifnum#1=3
    \fill (0.25,0.25) circle (0.1);
    \fill (0.5,0.5) circle (0.1);
    \fill (0.75,0.75) circle (0.1);
\fi
\ifnum#1=4
    \fill (0.25,0.25) circle (0.1);
    \fill (0.25,0.75) circle (0.1);
    \fill (0.75,0.25) circle (0.1);
    \fill (0.75,0.75) circle (0.1);
\fi
\ifnum#1=5
    \fill (0.25,0.25) circle (0.1);
    \fill (0.25,0.75) circle (0.1);
    \fill (0.5,0.5) circle (0.1);
    \fill (0.75,0.25) circle (0.1);
    \fill (0.75,0.75) circle (0.1);
\fi
\ifnum#1=6
    \fill (0.25,0.25) circle (0.1);
    \fill (0.25,0.5)  circle (0.1);
    \fill (0.25,0.75) circle (0.1);
    \fill (0.75,0.25) circle (0.1);
    \fill (0.75,0.5)  circle (0.1);
    \fill (0.75,0.75) circle (0.1);
\fi
\end{tikzpicture}%
}

\sloppy % Lockerer Umbruch, verhindert Überstand

\begin{document}
\begin{itemize}
\item The game is played with a total of six dice.
\item The goal of the game is to reach ten thousand points in as few rounds as possible.
\item A round begins with the first roll using all six dice.
\item After each roll, at least one scoring die must be set aside.
\item The value of individual dice and dice combinations is given in the table below.
\item If all dice have been successfully set aside in one or more rolls, the round can continue (again with all six dice).
\item A round ends when either the accumulated points are recorded or no points are scored in a roll.
\item Points may only be recorded starting from three hundred fifty points.
\item If a three-of-a-kind or a straight is rolled, it must be confirmed with at least three hundred fifty points. Before that, the accumulated points may not be recorded.
\end{itemize}
% Table generated by Excel2LaTeX from sheet 'Sheet1'
\begin{table}[H]
\centering
\begin{tabularx}{\adjustedWidth}{X}
\toprule
\di{5} = 50 points \\
\di{1} = 100 points \\
\midrule
3 x \di{2} = 200 points \\
3 x \di{3} = 300 points \\
3 x \di{4} = 400 points \\
3 x \di{5} = 500 points \\
3 x \di{6} = 600 points \\
3 x \di{1} = 1000 points \\
\midrule
Each additional die: \\
+ 1000 points \\
for example: \\
4 x \di{5} = 1500 points \\
5 x \di{3} = 2300 points \\
6 x \di{1} = 4000 points \\
\midrule
Three pairs \\
e.g.: \di{1} \di{1} \di{2} \di{2} \di{4} \di{4} \\
Straight (1 to 6) \\
e.g.: \di{1} \di{2} \di{3} \di{4} \di{5} \di{6} \\
= 1500 points \\
\bottomrule
\end{tabularx}%
\end{table}%

\end{document}
